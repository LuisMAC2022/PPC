\documentclass{IEEEtran}

\usepackage{xcolor}
\usepackage{amssymb}
\usepackage{tikz}
\usepackage{amsmath}

\title{Flujo de Trabajo}
\author{Elaborado por: LaRana}
\date{26 de Febrero de 2025}
\newcommand{\be}[2]{\begin{#1}{#2}\end{#1}}
\newcommand{\mb}[1]{\mathbb{#1}}
\newcommand{\ul}[1]{\underline{#1}}
\newcommand{\ol}[1]{\overline{#1}}
\newcommand{\mono}[4]{\be{center}{\be{tikzpicture}{ \pgfmathsetmacro{\Xbod }{#3 +.75} \pgfmathsetmacro{\Ybod }{#4 -1.3} \filldraw[fill = {#1} , thick] ({#3},{#4}) circle (0.5); \filldraw[fill={#2},thick](\Xbod,\Ybod  )  arc(0:180:0.8); } }}

\newcommand{\sumin}{\sum_{i=1}^n}
\newtheorem{teo}{Teorema}
\newtheorem{poof}{Demostraci\'{o}n}


\begin{document}
\maketitle
Los equipos permanentes (Administraci\'{o}n de Sistemas, QA\&Optimizaci\'{o}n, Seguridad \& Pruebas ) tienen 5 roles que cumplir: 

\be{itemize}{
	\item \textcolor{red}{Lider de equipo}
	\item \textcolor{green}{Responsable de Calidad}
	\item \textcolor{blue}{Responsable de Documentaci\'{o}n}
	\item \textcolor{purple}{Especialista}
	\item \textcolor{orange!90!black}{Auxiliar}
	}
Idealmente cada rol es cubierto por una persona.
%
\mono{blue}{blue}{0}{7}
\mono{green}{green}{2}{1}
De ser necesario una persona puede cubrir hasta dos roles.
\mono{red}{blue}{3}{4}
\mono{green}{purple}{1}{2}
Los roles de lider, responsables de calidad,  documentaci\'{o}n y especialista se espera que trabajen consistentemente en el mismo equipo durante la duraci\'{o}n del proyecto.
\mono{red}{red}{-2}{1}
El rol de auxiliar se espera que brinde apoyo en diferentes equipos, ya sea en los permanentes o en los complementarios. Dependiendo principalmente de la carga de trabajo.
\mono{orange!90!black}{orange!90!black}{0}{-1}
\newpage
Los equipos complementarios (Topolog\'{i}a,Redes, \\Automatizaci\'{o}n) estaran a cargo de un coordinador
\mono{yellow}{yellow}{-2}{1}
Los coordinadores se coordinan con los lideres de equipo, manager del proyecto y lider del proyecto para generar una teor\'{i}a unificada del sistema. 

\mono{black}{black}{-2}{1}
\mono{yellow}{yellow}{-2}{1}
\mono{red}{red}{-2}{1}

Los responsables de calidad y los responsables de documentaci\'{o}n conforman un equipo en si mismo para garantizar que el estilo y la calidad sean consistentes.

\mono{blue}{blue}{0}{7}
\mono{blue}{blue}{0}{7}

\mono{green}{green}{2}{1}
\mono{green}{green}{2}{1}

%\be{center}{
%	\be{tikzpicture}{
%		\filldraw[fill=blue, thick] (-0.25,1.3) circle (0.5);
%		%\draw[thick] (0,1) arc[start angle = 0, end angle =180, radius= 1](0.5);
%		\filldraw[fill = blue, thick] (0.5,0) arc(0:180:0.8);
%		\draw[thick] (-1.1,0) -- (0.5,0);
%			}
%}




%\input{carpeta/capitulo}
\end{document}
%\newtheorem{teo}{Teorema}

%b - brackets matrix/
% \[  \begin{bmatrix} 0 & 0 & 0\\ 0 & 0 & 0 \end{bmatrix} \]
%%v - vertical line matrix/
% \[  \begin{vmatrix} 0 & 0 & 0\\ 0 & 0 & 0 \end{vmatrix} \]
 %V - double vertical line  matrix/
%
% \[  \begin{Vmatrix} 0 & 0 & 0\\ 0 & 0 & 0 \end{Vmatrix} \]
%p - parenthesis matrix/
% \[  \begin{pmatrix} 0 & 0 & 0\\ 0 & 0 & 0 \end{pmatrix} \]

%\[ \left\{ \begin{array}{ccc}  \end{array}        \right\} \]
%
%
%
%
%
%
%
%







