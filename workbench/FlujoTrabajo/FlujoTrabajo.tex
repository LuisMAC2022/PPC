\documentclass{IEEEtran}


\usepackage{amssymb}
\usepackage{tikz}
\usepackage{amsmath}

\title{Flujo de Trabajo}
\author{Elaborado por: LaRana}
\date{26 de Febrero de 2025}
\newcommand{\be}[2]{\begin{#1}{#2}\end{#1}}
\newcommand{\mb}[1]{\mathbb{#1}}
\newcommand{\ul}[1]{\underline{#1}}
\newcommand{\ol}[1]{\overline{#1}}
\newcommand{\mono}[3]{\be{center}{\be{tikzpicture}{ \filldraw[fill = {#1} , thick] ({#2},{#3}) circle (0.5); \filldraw[fill={#1},thick]({#2}+.75,{#3 - 1.3} )  arc(0:180:0.8); \draw[thick] ({#2 -.85 },{#3-1.3})--(0.5,0); } }}

\newcommand{\sumin}{\sum_{i=1}^n}
\newtheorem{teo}{Teorema}
\newtheorem{poof}{Demostraci\'{o}n}


\begin{document}
\maketitle

\be{center}{
	\be{tikzpicture}{
		\filldraw[fill=blue, thick] (-0.25,1.3) circle (0.5);
		%\draw[thick] (0,1) arc[start angle = 0, end angle =180, radius= 1](0.5);
		\filldraw[fill = blue, thick] (0.5,0) arc(0:180:0.8);
		\draw[thick] (-1.1,0) -- (0.5,0);
			}
}

\mono{blue}{1}{2}
%\input{carpeta/capitulo}
\end{document}
%\newtheorem{teo}{Teorema}

%b - brackets matrix/
% \[  \begin{bmatrix} 0 & 0 & 0\\ 0 & 0 & 0 \end{bmatrix} \]
%%v - vertical line matrix/
% \[  \begin{vmatrix} 0 & 0 & 0\\ 0 & 0 & 0 \end{vmatrix} \]
 %V - double vertical line  matrix/
%
% \[  \begin{Vmatrix} 0 & 0 & 0\\ 0 & 0 & 0 \end{Vmatrix} \]
%p - parenthesis matrix/
% \[  \begin{pmatrix} 0 & 0 & 0\\ 0 & 0 & 0 \end{pmatrix} \]

%\[ \left\{ \begin{array}{ccc}  \end{array}        \right\} \]
%
%
%
%
%
%
%
%







